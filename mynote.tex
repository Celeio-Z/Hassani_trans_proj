\documentclass[lang=en,12pt,twoside]{textbook}
% ----------------------------- Math Font ------------------------------------- %
\usepackage[T1]{fontenc}
\usepackage{newtx,anyfontsize}
\DeclareSymbolFont{CMlargesymbols}{OMX}{cmex}{m}{n}
\let\sum\relax
\let\prod\relax
\DeclareMathSymbol{\sum}{\mathop}{CMlargesymbols}{"50}
\DeclareMathSymbol{\prod}{\mathop}{CMlargesymbols}{"51}
%% \infty
\DeclareSymbolFont{symbolsCM}{OMS}{cmsy}{m}{n}
\SetSymbolFont{symbolsCM}{bold}{OMS}{cmsy}{b}{n}
\let\txinfty\infty
\DeclareMathSymbol{\infty}{\mathord}{symbolsCM}{"31}
\definecolor{nuanbai}{HTML}{f5f5f5}
\pagecolor{lightgray!6}
% ---------------------------------------------------------------------------- %
% \overfullrule=5pt
% \showboxdepth=\maxdimen
% \showboxbreadth=\maxdimen
\tracingonline=1
\tracingoutput=1

\begin{document}
%----------------------------------------------------------------------------------------
%	标题页信息
%----------------------------------------------------------------------------------------
\title{Reading Paper}
\subtitle{\textit{Notes for Demailly's books and papers}}
\author{Ethan Lu}
\date{\today}
\publishers{Textbook}
%----------------------------------------------------------------------------------------

%----------------------------------------------------------------------------------------
%	插入自定义标题页
%----------------------------------------------------------------------------------------
\begin{titlepage} % 创建一个新的页面
%用来将图片从左下角开始平铺整个封面
	\AddToShipoutPicture*{%
	\AtPageLowerLeft{%
		\adjustbox{width=1.1\paperwidth, height=1.5\paperheight, keepaspectratio}{% 强制图片至纸张尺寸,但可能裁切
			\includegraphics{images/pexels-photo-3378916.jpeg}
		}
	}
}
\begin{flushleft} % 左对齐环境
	\setlength{\leftskip}{1cm} % 左侧缩进
	\makeatletter % 允许访问带有@字符的内部命令
	% \vspace*{4cm} % 标题距离页面顶端的空白
	% {\color{white}\Huge \textbf{\@title} \par} % 使用前文定义的title作为标题
	% \vspace{1cm} % 标题和子标题的间距
	% {\color{white}\Large \@subtitle \par} % 使用前文定义的subtitle作为子标题
	% \vfill % 作者信息前的垂直填充
	% {\color{white}\large \@author \par} % 作者名
	% {\color{white}\large \@publishers \par} % 出版者
	% {\color{white}\large \today\par} % 日期,默认为当天
        \begin{tikzpicture}[overlay,remember picture]
        \begin{pgfonlayer}{bottom}
            \fill[dblue!10,opacity=0.1] (current page.south west) rectangle ++(\paperwidth,2cm);
            \node[inner sep=0pt,text=white,font=\large\sffamily,above] (bottominfo) at ([yshift=.7cm]current page.south) {
                \@author\hspace{4cm}\@publishers\hspace{4cm}\today
            };
        \end{pgfonlayer}
        \fill[color=black!50,opacity=.2]node[append after command={
            ([yshift=0.5cm]bookinfo.north west) rectangle ([yshift=-0.5cm]bookinfo.south east)},minimum width=\paperwidth,opacity=1,align=left,inner sep=0pt,anchor=west] (bookinfo) at ([shift={(0,4cm)}]current page.west) {\hspace{-7cm}
                \begin{varwidth}{\linewidth}
                    \setlength\baselineskip{3ex}
                    \textcolor{black!10!white}{\Huge \textbf{\@title}} \\[.6cm]
                    \textcolor{black!10!white}{\Large \@subtitle}
                \end{varwidth}
                };
    \end{tikzpicture}
	\makeatother % 将@重置为非字母字符
	\vspace{0cm} % 标题和子标题的间距
	% 结束左对齐环境
\end{flushleft}
\ClearShipoutPicture % 清除背景图片
\end{titlepage}
% --------------------------------- 主要内容写在下面 --------------------------------- %
\pagestyle{Mainpage} % 页面样式
\chapimg{images/pexels-photo-1452701.jpeg}

\begin{titlepage}
    \newgeometry{left=2cm,right=2cm,top=2.5cm,bottom=2.2cm}
    \tableofcontents
    \restoregeometry
\end{titlepage}

% ---------------------------------------------------------------------------- %

\partimg{images/pexels-photo-931018.jpeg}
\part{Proof of Opennes Conjecture}


\chapter{Notes for the Proof of Openness Conjecture}

\section{Organizing the proof strategy}
\subsection{The case \texorpdfstring{$1$}{}}

\begin{lemma}[Lemma 2.3 (see [8]; see also [22]).]\label{lem:2.3}
     Let $F \in \mathcal{O}_n$ and $g_1, \ldots, g_s \in \mathcal{O}_n$ be germs of holomorphic functions vanishing at the origin $o \in \mathbb{C}^n$. Assume that for any given neighborhood of $o,|F| \leq C\left|\left(g_1, \ldots, g_s\right)\right|$ does not hold for any constant $C$. Then there exists a germ of an analytic curve $\gamma$ through o, satisfying $\gamma \cap\{F=0\} \subseteq\{o\}$, such that $\left.\frac{g_i}{F}\right|_\gamma$ is holomorphic on $\gamma \backslash o$ with \ref{thm:1.1} for any $i \in\{1, \ldots, s\}$, where $\frac{\widetilde{g_i}}{F}$ is the holomorphic extension of $\frac{g_i}{F}$ from $\gamma \backslash o$ to $\gamma$.
\end{lemma}

\begin{theorem}[cf \cite{guan2015proof}]\label{thm:1.1}
    Let $\varphi$ be a negative plurisubharmonic function on the unit polydisc
    $\Delta^n \in \mathbb{C}^n$. Suppose $F$ is a holomorphic function on
    $\Delta^n$, which satisfies
    \[ \int_{\Delta^n} | F |^2 e^{- \varphi} d \lambda_n < + \infty, \]
    where $d \lambda_n$ is the Lebesgue measure on $\mathbb{C}^n$. Then for some
$r \in (0, 1)$, there exists a number $p > 1$ such that
\[ \int_{\Delta^n_r} | F |^2 e^{- p \varphi} d \lambda_n < + \infty . \]
\end{theorem}

\begin{proof}
As
\[ \int_{\Delta' \times \Delta''} | F_j |^2 e^{- p_j \varphi} d \lambda_n
    \leqslant C \int_{H_j} | F |^2 e^{- p_j \varphi} d \lambda_{n - 1} \]
and we already known that\sidenote{see (3.7) in original paper.}
\[ \int_{H_j} | F |^2 e^{- p_j \varphi} d \lambda_{n - 1} \leqslant 2\int_{H_j}
    | F |^2 e^{- \varphi} d \lambda_{n - 1} = o \left( \frac{1}{| a_j  |^2}
    \right), \]
thus we have
\begin{align*}
\int_{\Delta' \times \Delta''} | F_j |^2 e^{- p_j \varphi} d \lambda_n &
\leqslant  C \int_{H_j} | F |^2 e^{- p_j \varphi} d \lambda_{n - 1}\\
& \leqslant  C \int_{H_j} | F |^2 e^{- \varphi} d \lambda_{n - 1} = o
\left( \frac{1}{| a_j  |^2} \right) .
\end{align*}
Then by $F_j |_{H_j} = F |_{H_j}$ and negativeness of $\varphi$, we have
\[ \int_{\Delta' \times \Delta''} | F_j |^2 d \lambda_n \leqslant
    \int_{\Delta' \times \Delta''} | F_j |^2 e^{- p_j \varphi} d \lambda_n = o
    \left( \frac{1}{| a_j  |^2} \right), \]
which is to say
\[ \int_{\Delta' \times \Delta''} | F_j |^2 d \lambda_n = o \left( \frac{1}{|
    a_j  |^2} \right) . \]
\tikz\draw[line width=0.5pt,dashed,myblue] (0,0)--++(.95\linewidth,0);\newline
We choose $r_0$ small enough such that\sidenote{The reason is that from the assumption of theorem \ref{thm:1.1}, we have known that
$F$ is a holomorphic function on $\Delta^n$, which satisfies
\[ \int_{\Delta^n} | F |^2 e^{- \varphi} d \lambda_n < + \infty, \]
thus we can find a small enough $r_0$ such that
\[ \int_{\Delta_{r_0}} | F |^2 e^{- \varphi} d \lambda_1 < + \infty . \]}
\[ \int_{\Delta_{r_0}} | F |^2 e^{- \varphi} d \lambda_1 < + \infty . \]


% \softclearmargin
\textit{Step 1: Theorem \ref{thm:1.1} for the dimension one case.}
We first prove Theorem \ref{thm:1.1} for the dimension one case, which is elementary but instructive. Our proof for the general dimension case is quite similar. Actually the proofs for both cases are parallel.

We choose $r_0$ small enough such that $\int_{\Delta_{r_0}}|F|^2 e^{-\varphi} d \lambda_1<+\infty$\sidenote{It has already been explained above.}. Then there exist complex numbers $a_j \rightarrow 0 \; (j \rightarrow+\infty)$ such that
$$
\left|F\left(a_j\right)\right|^2 e^{-\varphi\left(a_j\right)}=o\left(\frac{1}{\left|a_j\right|^2}\right) .
$$

As $\left|F\left(a_j\right)\right|^2 e^{-\varphi\left(a_j\right)}<+\infty$, then one can find $p_j>1$ small enough such that\sidenote{As \begin{align*}
    &\phantom{=}e^{- p_j \varphi (a_j)} \\
    &= [e^{- \varphi (a_j)}]^{p_j} \\ &= e^{- \varphi (a_j)}
       \cdot [e^{- \varphi (a_j)}]^{p_j - 1},
\end{align*}
$p_j > 1$ ($p_j - 1 > 0$) and the negativeness of $\varphi$ ($-
\varphi > 0$), there exists $p_j$ small enough such that $p_j - 1 \leqslant
\ln 2$, thus we have
\[ [e^{- \varphi (a_j)}]^{p_j - 1} \leqslant e^{p_j - 1} \leqslant 2. \]}

\begin{equation}\label{eq:3.1}
    \left|F\left(a_j\right)\right|^2 e^{-p_j \varphi\left(a_j\right)} \leq 2\left|F\left(a_j\right)\right|^2 e^{-\varphi\left(a_j\right)}=o\left(\frac{1}{\left|a_j\right|^2}\right) .
\end{equation}


Using movable (respect to $a_j$ ) the Ohsawa-Takegoshi $L^2$ extension theorem on $\Delta$, we obtain holomorphic functions $F_j$ on $\Delta$ such that $\left.F_j\right|_{a_j}=F\left(a_j\right)$ and\sidenote{This is because $| F (a_j) |^2\cdot$ $ e^{- p_j (a_j)}$ is a constant.}
\begin{equation}\label{eq:3.2}
    \int_{\Delta}\left|F_j\right|^2 e^{-p_j \varphi} d \lambda_1 \leq \mathbf{C}\left|F\left(a_j\right)\right|^2 e^{-p_j \varphi\left(a_j\right)},
\end{equation}
where $\mathbf{C}$ is a universal constant.

By inequality \eqref{eq:3.1} and negativeness of $\varphi$, we obtain that
\begin{equation}\label{eq:3.3}
    \int_{\Delta}\left|F_j\right|^2 d \lambda_1=o\left(\frac{1}{\left|a_j\right|^2}\right) .
\end{equation}

By contradiction, assume that Theorem \ref{thm:1.1} does not hold for $n=1$; that is to say, $\int_{\Delta_r}|F|^2 e^{-p_j \varphi} d \lambda=+\infty$ for any $r>0$ and any $j \in\{1,2, \ldots\}$.

\textbf{Assertion.} Since $\{F=0\} \cap \Delta_{r_0} \subseteq\{o\}$, then it follows from inequality \eqref{eq:3.2} that one can derive that \emph{$F / F_j$ is unbounded}. Otherwise, the boundedness would imply the finiteness of the integral of $|F|^2 e^{-p_j \varphi}$, according to inequality \eqref{eq:3.2}. This contradicts the assumption. Then there exists a holomorphic function $h_j$ on $\Delta_{r_0}$ satisfying
\begin{enumerate}[label=(\arabic*)]
    \item $\left.F_j\right|_{\Delta_{r_0}}=\left.F\right|_{\Delta_{r_0}} h_j$,
    \item $h_j(o)=0$,
    \item $h_j\left(a_j\right)=1$.
\end{enumerate}
According to Lemma 2.1\sidenote{Let $f \not \equiv 0$ be a holomorphic function on the disc $\Delta_r$ of radius $r$ containing the origin o in $\mathbb{C}$. Let $h_a$ be a holomorphic function on $\Delta_r$, which satisfies $h_a(o)=0$ and $h_a(b)=1$ for any $b^k=a$ ( $k$ is a positive integer), where $a \in \Delta_r$ whose norm is small enough. Then we have
$$
\int_{\Delta_r}|f|^2\left|h_a\right|^2 d \lambda_1>C_1|a|^{-2}
$$
where $C_1$ is a positive constant independent of a and $h_a$.}, it follows that
$$
\frac{C_1}{\left|a_j\right|^2} \leq \int_{\Delta_{r_0}}\left|F_j\right|^2 d \lambda_1,
$$
which contradicts equality \eqref{eq:3.3}, where $C_1>0$ is independent of $j$.
We have thus proved Theorem \ref{thm:1.1} for $n=1$.

\textit{Step 2: Theorem \ref{thm:1.1} for $n$.} By induction on the dimension $n$, one may assume that Theorem \ref{thm:1.1} holds for $n-1$.

We prove Theorem \ref{thm:1.1} for the general dimension $n$ by contradiction. Assume that Theorem \ref{thm:1.1} for $n$ is not true, therefore, for some negative psh function $\varphi$, there exists a holomorphic function $F$ such that
\begin{equation}\label{eq:3.4}
    \int_{\Delta_{r_0}^n}|F|^2 e^{-\varphi} d \lambda_n<+\infty,
\end{equation}
for some $r_0>0$, and
\begin{equation}\label{eq:3.5}
    \int_{\Delta_r^n}|F|^2 e^{-p \varphi} d \lambda_n=+\infty,
\end{equation}
for any $r \in\left(0, r_0\right)$ and $p>1$. That is to say, the germ of the holomorphic function $F$ is in $\mathcal{I}(\varphi)_o$ but not in $\mathcal{I}_{+}(\varphi)_o$.

By Lemma 2.4\sidenote{Assume that $F \in \mathcal{O}_n$ is a holomorphic function on some neighborhood $V$ of $o$ that is not a germ of $$\mathcal{I}\left(\psi_{j_1}\right)_o=\left(\cup_{j=1}^{\infty} \mathcal{I}\left(\psi_j\right)\right)_o.$$ Then there exists a germ of an analytic curve $(\gamma, o)$ such that $$\left.\frac{\widetilde{g o \gamma}}{F o \gamma}\right|_o=0$$ holds for any germ $(g, o)$ of $\mathcal{I}\left(\psi_{j_1}\right)_o$, where $\frac{\widetilde{g \circ \gamma}}{F \circ \gamma}$ is the holomorphic extension of $\frac{g \circ \gamma}{F \circ \gamma}$ from $\gamma \backslash o$ to $\gamma$.} and equality \ref{eq:3.5}, it follows that there exists a germ of an analytic curve $\gamma$ through $o$ satisfying $\left\{\left.F\right|_\gamma=0\right\} \subseteq\{o\}$ such that for any germ of holomorphic function $g$ in $\mathcal{I}_{+}(\varphi)_o$, there exists a holomorphic function $h_g$ on $\gamma$ satisfying
\begin{equation}\label{eq:3.6}
    h_g(o)=0 \quad \text { and }\left.\quad g\right|_\gamma=\left.F\right|_\gamma h_g .
\end{equation}

This plays a similar role with the assertion in the proof for dimension $1$.

Using the local parametrization of $\gamma$ (see [5]), without loss of generality, one may assume that $\gamma$ and $\Delta^{\prime} \times \Delta^{\prime \prime}$ are as in Section 2.1.

By inequality \eqref{eq:3.4}, it follows that there exist hyperplanes $H_j:=H_{a_j}=$ $\left\{z^{\prime}=a_j\right\}$ that satisfy $a_j \rightarrow 0(j \rightarrow \infty)$ and

\begin{equation}\label{eq:convergence}
    \int_{H_j}|F|^2 e^{-\varphi} d \lambda_{n-1}=o\left(\frac{1}{\left|a_j\right|^2}\right) .
\end{equation}


According to \textit{the induction assumption} and \textit{the Lebesgue dominated convergence theorem}, it follows that for any given $j$, there exists a small enough $p_j>1$ such that\sidenote{It may be the reason that from the truth of one case in \eqref{eq:3.1}, we can obtain that when $a_j\to 0$, we have \eqref{eq:convergence}, thus for any given $j$, by dominated convergence theorem, \eqref{eq:3.7} is true.}
\begin{equation}\label{eq:3.7}
    \int_{H_j}|F|^2 e^{-p_j \varphi} d \lambda_{n-1} \leq 2 \int_{H_j}|F|^2 e^{-\varphi} d \lambda_{n-1}=o\left(\frac{1}{\left|a_j\right|^2}\right) .
\end{equation}

Using movably (respect to $j \rightarrow \infty$ ) \textit{the Ohsawa-Takegoshi $L^2$ extension theorem} on $\Delta^{\prime} \times \Delta^{\prime \prime}$, we obtain a holomorphic function $F_j:=F_{a_j}$ on $\Delta^{\prime} \times \Delta^{\prime \prime}$ for each $j$ such that $\left.F_j\right|_{H_j}=\left.F\right|_{H_j}$, and
$$
\int_{\Delta^{\prime} \times \Delta^{\prime \prime}}\left|F_j\right|^2 e^{-p_j \varphi} d \lambda_n \leq \mathbf{C} \int_{H_j}|F|^2 e^{-p_j \varphi} d \lambda_{n-1},
$$
where $\mathbf{C}$ is a universal constant.
By inequality \eqref{eq:3.7} and negativeness of $\varphi$, it follows that
\begin{equation}\label{eq:3.8}
    \int_{\Delta^{\prime} \times \Delta^{\prime \prime}}\left|F_j\right|^2 d \lambda_n=o\left(\frac{1}{\left|a_j\right|^2}\right)
\end{equation}

Note that $\left.F_j\right|_{H_j}=\left.F\right|_{H_j}$ and $\left(F_j, o\right) \in \mathcal{I}_{+}(\varphi)_o$ but $(F, o) \notin \mathcal{I}_{+}(\varphi)_o$. According to equality \eqref{eq:3.6} and Lemma 2.2\sidenote{Let $F$ be a holomorphic function on $\Delta^{\prime} \times \Delta^{\prime \prime}$ satisfying $\left.F\right|_\gamma \not \equiv 0$. Denote by hyperplane $H_a:=\left\{z^{\prime}=a\right\}$ near o. Let $F_a$ be the holomorphic extension of $\left.F\right|_{H_a}$ on $\Delta^{\prime} \times \Delta^{\prime \prime}$ such that there exists a holomorphic function $h_a$ on $\gamma$ satisfying
\begin{enumerate}[label=(\arabic*)]
    \item $\left.F_a\right|_\gamma=\left.F\right|_\gamma h_a$,
    \item $h_a(o)=0$.
\end{enumerate}
Then we have
$$
\int_{\Delta^{\prime} \times \Delta^{\prime \prime}}\left|F_a\right|^2 d \lambda_n \geq \frac{C_2}{|a|^2}
$$
where $C_2$ is a positive constant independent of $a, H_a$ and $F_a$.}, it follows that
$$
\int_{\Delta^{\prime} \times \Delta^{\prime \prime}}\left|F_j\right|^2 d \lambda_n \geq \frac{C_2}{\left|a_j\right|^2}
$$
where $C_2>0$ is independent of $j$, which contradicts equality \eqref{eq:3.8}.
We have thus proved Theorem \ref{thm:1.1} for $n$. The proof of Theorem \ref{thm:1.1} is now complete.


\end{proof}


% \begin{figure}[!htpb]
%     \centering
%     \subfigure[0.25\textwidth][fig2test\label{fig1}]{
%     \includegraphics[width=.45\textwidth]{example-image}}
%     \hspace{.5cm} % Adjust the space as needed
%     \subfigure[0.25\textwidth][fig2test\label{fig2}]{
%     \includegraphics[width=.45\textwidth]{example-image}}
%     \caption{fig}
%     \label{fig}
%     \end{figure}

%     \ref{fig} big; small1 \ref{fig1} and small2 \ref{fig2}
\section{Some topics may be benifit for the future research}
\subsection{Multiplier ideal sheaf \texorpdfstring{$\mJ(\varphi)$}{}}
\begin{definition}[Multiplier ideal sheaf]\label{defi:MIF}
    Multiplier ideal sheaf $\mJ(\varphi)$ is considered as the subsheaf of germs of holomorphic functions $f$  such that $\abs{f}^2 e^{-2\varphi}$ is locally integrable (/summable).

    It is worth to point out that it is a \emph{coherent algebraic sheaf} over $X$ and satisfies
    \[H^q(X,K_X\otimes L\otimes \mJ(\varphi))=0\]
    for all $q\geq 1$ if the curvature of $L$ is positive in the sense of current.
\end{definition}

The two efficient ways to solve the singularities are \emph{Multiplier ideal sheaf} by Nadel (Which can be seen as a generalization of Kawamata-Viehweg's vanishing theorem.) and \emph{The theory of positive currents} by Lelong. Currents can be seen as generalization of \textbf{algebraic cycles}.

\subsection{Plurisubharmonic functions}
\begin{definition}[Plurisubharmonic functions]\label{def:psh functions}
    A function $u: \Omega \longrightarrow\left[-\infty,+\infty\left[\right.\right.$ defined on an open subset $\Omega \subset \mathbb{C}^n$ is said to be plurisubharmonic (psh for short) if
    \begin{enumerate}[label=(\alph*)]
        \item $u$ is upper semicontinuous;
        \item for every complex line $L \subset \mathbb{C}^n$, $u_{\uparrow \Omega \cap L}$ is subharmonic on $\Omega \cap L$, that is, for all $a \in \Omega$ and $\xi \in \mathbb{C}^n$ with $|\xi|<d(a, \complement \Omega)$, the function $u$ satisfies the mean value inequality\sidenote{A vital characteristic, which is equivalent to the integrability of the inegral.}
$$
u(a) \leqslant \frac{1}{2 \pi} \int_0^{2 \pi} u\left(a+e^{\mathrm{i} \theta} \xi\right) d \theta.
$$
    \end{enumerate}
The set of psh functions on $\Omega$ is denoted by $\operatorname{Psh}(\Omega)$.
\end{definition}

\begin{theorem}[]\label{thm:psh-criterion}
    A function $u\in C^2(\Omega,\bR)$ is psh if and only if the hermitian form
    \begin{equation*}
        Hu(a)(\xi)=\sum_{1\leqslant j,k\leqslant n}\frac{\bd^2 u}{\bd z_j\bd\bar z_k(a)} \xi_j\bar\xi_k
    \end{equation*}
    is semi-positive at every point $a\in \Omega$.
\end{theorem}

\subsection{Positive Currents}
A current of degree $q$ on an oriented differentiable manifold $M$ is simply a differential $q$-form $\Theta$ with distribution coefficients. The space of currents of degree $q$ over $M$ will be denoted by $\mathscr{D}^{\prime q}(M)$. Alternatively, a current of degree $q$ can be seen as an element $\Theta$ in the dual space $\mathscr{D}_p^{\prime}(M):=\left(\mathscr{D}^p(M)\right)^{\prime}$ of the space $\mathscr{D}^p(M)$ of smooth differential forms of degree $p=\operatorname{dim} M-q$ with compact support; the duality pairing is given by
\begin{equation}\label{eq:2.1}
    \langle\Theta, \alpha\rangle=\int_M \Theta \wedge \alpha, \quad \alpha \in \mathscr{D}^p(M) .
\end{equation}
A basic example is the current of integration $[S]$ over a compact oriented submanifold $S$ of $M$ :
\begin{equation}\label{eq:2.2}
    \langle[S], \alpha\rangle=\int_S \alpha, \quad \operatorname{deg} \alpha=p=\operatorname{dim}_{\mathbb{R}} S
\end{equation}
Then $[S]$ is a current with measure coefficients, and Stokes' formula shows that $d[S]=$ $(-1)^{q-1}[\partial S]$, in particular $d[S]=0$ if $S$ has no boundary. Because of this example, the integer $p$ is said to be the dimension of $\Theta$ when $\Theta \in \mathscr{D}_p^{\prime}(M)$. The current $\Theta$ is said to be closed if $d \Theta=0$.

% \softclearmargin
On a complex manifold $X$, we have similar notions of bidegree and bidimension; as in the real case, we denote by
\begin{equation}\label{eq:2.3}
    \mathscr{D}^{\prime p, q}(X)=\mathscr{D}_{n-p, n-q}^{\prime}(X), \quad n=\operatorname{dim} X,
\end{equation}
the space of currents of bidegree $(p, q)$ and bidimension $(n-p, n-q)$ on $X$. According to \cite{lelong1957integration}\marginpar{\small \cite{lelong1957integration} \fullcite{lelong1957integration}}, a current $\Theta$ of bidimension ( $p, p$ ) is said to be (weakly) positive if for every choice of smooth ( 1,0 )-forms $\alpha_1, \ldots, \alpha_p$ on $X$ the distribution
$$\Theta \wedge (\mathrm{i} \alpha_1 \wedge \bar{\alpha}_1) \wedge \ldots \wedge (\mathrm{i} \alpha_p \wedge \bar{\alpha}_p) $$ is a positive measure.

\section{Perverse Sheaf and Intersection Cohomology}

\begin{itemize}
    \item test
    \begin{itemize}
        \item testing
        \begin{itemize}
            \item tested
        \end{itemize}
        \item testii
    \end{itemize}
    \item test2
    \item test3
\end{itemize}

\begin{enumerate}
    \item test
    \begin{enumerate}
        \item testing
        \begin{enumerate}
            \item tested
        \end{enumerate}
        \item testii
    \end{enumerate}
    \item test2
    \item test3
\end{enumerate}

\chapter{test}
\section{test1}
\section{test2}
\chapter*{test}


\chapter*{Test the long title}
















\newgeometry{left=2cm,right=2cm,bottom=2cm,top=3cm}
%
\printbibliography[heading=bibintoc]
 %
\end{document}